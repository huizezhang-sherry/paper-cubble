\documentclass{article}

\usepackage{arxiv}

\usepackage[utf8]{inputenc} % allow utf-8 input
\usepackage[T1]{fontenc}    % use 8-bit T1 fonts
\usepackage{lmodern}        % https://github.com/rstudio/rticles/issues/343
\usepackage{hyperref}       % hyperlinks
\usepackage{url}            % simple URL typesetting
\usepackage{booktabs}       % professional-quality tables
\usepackage{amsfonts}       % blackboard math symbols
\usepackage{nicefrac}       % compact symbols for 1/2, etc.
\usepackage{microtype}      % microtypography
\usepackage{graphicx}

\title{A template for the \emph{arxiv} style}

\author{
    H.Sherry Zhang
   \\
    Department of Econometrics and Business Statistics \\
    Monash University \\
  Melbourne, Australia \\
  \texttt{\href{mailto:huize.zhang@monash.edu}{\nolinkurl{huize.zhang@monash.edu}}} \\
   \And
    Dianne Cook
   \\
    Department of Econometrics and Business Statistics \\
    Monash University \\
  Melbourne, Australia \\
  \texttt{\href{mailto:dicook@monash.edu}{\nolinkurl{dicook@monash.edu}}} \\
   \And
    Ursula Laa
   \\
    Institute of Statistics \\
    University of Natural Resources and Life Sciences \\
  Vienna, Austria \\
  \texttt{\href{mailto:ursula.laa@boku.ac.at}{\nolinkurl{ursula.laa@boku.ac.at}}} \\
   \And
    Nicolas Langrené
   \\
    34 Village Street, Docklands VIC 3008 Australia \\
    CSIRO Data61 \\
  Melbourne, Australia \\
  \texttt{\href{mailto:nicolas.langrene@csiro.au}{\nolinkurl{nicolas.langrene@csiro.au}}} \\
   \And
    Patricia Menéndez
   \\
    Department of Econometrics and Business Statistics \\
    Monash University \\
  Melbourne, Australia \\
  \texttt{\href{mailto:patricia.menendez@monash.edu}{\nolinkurl{patricia.menendez@monash.edu}}} \\
  }



% Pandoc citation processing
\newlength{\csllabelwidth}
\setlength{\csllabelwidth}{3em}
\newlength{\cslhangindent}
\setlength{\cslhangindent}{1.5em}
% for Pandoc 2.8 to 2.10.1
\newenvironment{cslreferences}%
  {}%
  {\par}
% For Pandoc 2.11+
\newenvironment{CSLReferences}[2] % #1 hanging-ident, #2 entry spacing
 {% don't indent paragraphs
  \setlength{\parindent}{0pt}
  % turn on hanging indent if param 1 is 1
  \ifodd #1 \everypar{\setlength{\hangindent}{\cslhangindent}}\ignorespaces\fi
  % set entry spacing
  \ifnum #2 > 0
  \setlength{\parskip}{#2\baselineskip}
  \fi
 }%
 {}
\usepackage{calc} % for calculating minipage widths
\newcommand{\CSLBlock}[1]{#1\hfill\break}
\newcommand{\CSLLeftMargin}[1]{\parbox[t]{\csllabelwidth}{#1}}
\newcommand{\CSLRightInline}[1]{\parbox[t]{\linewidth - \csllabelwidth}{#1}\break}
\newcommand{\CSLIndent}[1]{\hspace{\cslhangindent}#1}



\begin{document}
\maketitle

\def\tightlist{}


\begin{abstract}
Enter the text of your abstract here.
\end{abstract}

\keywords{
    blah
   \and
    blee
   \and
    bloo
   \and
    these are optional and can be removed
  }

\hypertarget{introduction}{%
\section{Introduction}\label{introduction}}

Saptio-temporal data

The rest of the paper will be divided as follows: Section 2 reviews the
existing data structure for spatio, temporal, and spatio-temporal data.
Section 3 presents a new data structure for spatio-temporal data:
cubble. Then the paper introduces the workflow of data manipulation and
visualisation with the cubble structure in Section 4. Section 5 gives
some examples on how common spatial and temporal manipulations are
performed with cubble and how static and interactive visualisation help
ot understand climate and {[}\ldots{]} data.

\hypertarget{existing-data-structure-for-spatio-and-temporal-data}{%
\section{Existing data structure for spatio and temporal
data}\label{existing-data-structure-for-spatio-and-temporal-data}}

There has been a large class of implementations dedicated to processing
the spatial data. This includes \texttt{sf} (E. J. Pebesma 2018) and its
precedent \texttt{sp} (E. Pebesma and Bivand 2005) for \ldots{} and
\texttt{raster} (Hijmans 2020) and \texttt{terra} (Hijmans 2021) for
raster data. While these implementations specialised in geographic
manipulations with different type of simple features, it doesn't
incorporate a temporal dimension in the data structure. Project like
\texttt{stars} (E. Pebesma 2021) and \texttt{spacetime} (Bivand,
Pebesma, and Gomez-Rubio 2013) by R-Spatial allows for both space and
time dimension for raster and vector data, but the underlying data
structure is a multi-dimensional array, which could be difficult to
operate for R users who are more familiar with the operation in 2D
dataframe/ tibble.

In the temporal aspect, the \texttt{tsibble} (Wang, Cook, and Hyndman
2020) structure and its tidyverts ecosystem have provided a {[}\ldots{}
{]} workflow to work with temporal data. In a tsibble structure,
temporal data is characterised by \texttt{index} and \texttt{key} where
\texttt{index} is the temporal identifier and \texttt{key} is the
identifier for multiple series, which could be used as a spatio
identifier. However, a tsibble object, by construction, always requires
the \texttt{index} in its structure. This makes it less appealing for
spatio-temporal data since the output of calculated spatio-specific
variables (i.e.~features of each series) don't have the time dimension.
Analysts will either need to have an additional step to join this output
to the original tsibble or operate with variables stored in two separate
objects. In addition, the long form structure of a tsibble object means
spatio variables (i.e.~longitude, latitude, and features of each series
if joined back to the tsibble) of each spatio identifier will be
repetitively recorded at each timestamp. This repetition is unnecessary
and would inflate the object size for long series.

\hypertarget{a-new-data-structure-for-spatio-temporal-data}{%
\section{A new data structure for spatio-temporal
data}\label{a-new-data-structure-for-spatio-temporal-data}}

Intro to cubble:

\begin{itemize}
\tightlist
\item
  list-column: rowwise\_df with temporal variables, including the time
  index, nested.

  \begin{itemize}
  \tightlist
  \item
    Focus on spatio: those output per station
  \end{itemize}
\item
  long form: grouped\_df

  \begin{itemize}
  \tightlist
  \item
    Focus on temporal
  \end{itemize}
\end{itemize}

Compatible with tidyverse manipulation and tsibble

\hypertarget{manipulation-and-visualisation-with-cubble}{%
\section{\texorpdfstring{Manipulation and visualisation with
\texttt{cubble}}{Manipulation and visualisation with cubble}}\label{manipulation-and-visualisation-with-cubble}}

Mention different types of manipulation with cubble:

\begin{itemize}
\tightlist
\item
  \texttt{dplyr} support for cubble:

  \begin{itemize}
  \tightlist
  \item
    basic 5s: mutate, filter, summarise, select, arrange
  \item
    group and ungorup: group\_by, ungroup
  \item
    slice family
  \end{itemize}
\item
  summarise missing stats
\end{itemize}

\hypertarget{examples}{%
\section{Examples}\label{examples}}

Daily climate data (prcp, tmax, and tmin) from RNOAA - lots of stations
across Australia

An exploratory data analysis questions: What's the climate profile look
like in Australia

\begin{itemize}
\tightlist
\item
  General features: Any general trend/ fluctuation in prcp, tmax, and
  tmin?
\item
  Local features: Any station stands out from the crowd?
\end{itemize}

\hypertarget{manipulation}{%
\subsection{Manipulation}\label{manipulation}}

\begin{itemize}
\tightlist
\item
  data quality check: filter out stations have variables not properly
  recorded
\item
  data summary:

  \begin{itemize}
  \tightlist
  \item
    daily -\textgreater{} monthly/ weekly,
  \item
    summarise by mean for tmax/ tmin, sum for prcp
  \end{itemize}
\item
\end{itemize}

\hypertarget{graphics}{%
\subsection{Graphics}\label{graphics}}

Static + interactive -\textgreater{} tooltip to show additional
information upon hovering

\begin{itemize}
\tightlist
\item
  Where are those stations on the map?

  \begin{itemize}
  \tightlist
  \item
    Mention mostly aero, airport, and lighthouse
  \end{itemize}
\end{itemize}

\hypertarget{summary}{%
\section*{Summary}\label{summary}}
\addcontentsline{toc}{section}{Summary}

\hypertarget{refs}{}
\begin{CSLReferences}{1}{0}
\leavevmode\hypertarget{ref-spacetimebook}{}%
Bivand, Roger S., Edzer Pebesma, and Virgilio Gomez-Rubio. 2013.
\emph{Applied Spatial Data Analysis with {R}, Second Edition}. Springer,
NY. \url{https://asdar-book.org/}.

\leavevmode\hypertarget{ref-raster}{}%
Hijmans, Robert J. 2020. \emph{Raster: Geographic Data Analysis and
Modeling}. \url{https://CRAN.R-project.org/package=raster}.

\leavevmode\hypertarget{ref-terra}{}%
---------. 2021. \emph{Terra: Spatial Data Analysis}.
\url{https://CRAN.R-project.org/package=terra}.

\leavevmode\hypertarget{ref-stars}{}%
Pebesma, Edzer. 2021. \emph{Stars: Spatiotemporal Arrays, Raster and
Vector Data Cubes}. \url{https://CRAN.R-project.org/package=stars}.

\leavevmode\hypertarget{ref-pebesma2018simple}{}%
Pebesma, Edzer J. 2018. {``Simple Features for r: Standardized Support
for Spatial Vector Data.''} \emph{R J.} 10 (1): 439.

\leavevmode\hypertarget{ref-pebesma2005s}{}%
Pebesma, Edzer, and Roger S Bivand. 2005. {``S Classes and Methods for
Spatial Data: The Sp Package.''} \emph{R News} 5 (2): 9--13.

\leavevmode\hypertarget{ref-tsibbles}{}%
Wang, Earo, Dianne Cook, and Rob J Hyndman. 2020. {``A New Tidy Data
Structure to Support Exploration and Modeling of Temporal Data.''}
\emph{Journal of Computational and Graphical Statistics} 29 (3):
466--78. \url{https://doi.org/10.1080/10618600.2019.1695624}.

\end{CSLReferences}

\bibliographystyle{unsrt}
\bibliography{references.bib}


\end{document}
